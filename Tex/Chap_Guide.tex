\chapter{Darknet移植}
\section{Darknet介绍}
Darknet是一个开源的C语言神经网络框架,由著名的机器视觉专家Joseph Redmon开发\cite{darknet13}。Darknet性能优异,安装简单,在目标识别领域有广泛应用的YOLO模型\cite{yolov3}的最初版本就是在这个框架下开发的。作为作为“神威 · 太湖之光”并行编程的入门练习,Darknet有如下优点\cite{darknet_note}:
\begin{itemize}
	\item 易于安装:在makefile里面选择自己需要的附加项(cuda,cudnn,opencv等)直接make即可;
	\item 没有任何依赖项:整个框架都用C语言进行编写,可以不依赖任何库,连opencv作者都编写了可以对其进行替代的函数;
	\item 结构清晰:其框架的基础文件都在src文件夹,而定义的一些检测、分类函数则在example文件夹,可根据需要直接对源代码进行查看和修改;
	\item 友好python接口:虽然darknet使用c语言进行编写,但是也提供了python的接口,通过python函数,能够使用python直接对训练好的.weight格式的模型进行调用;
	\item 易于移植:该框架部署十分简单,且可以根据机器情况,使用cpu和gpu,特别是检测识别任务的本地端部署,darknet会显得异常方便。
\end{itemize}

此外,由于Darknet是一个神经网络框架,其中的主要运算任务都可以以矩阵计算的形式所表述,这使得Darknet的移植和优化有助于读者对矩阵和并行更加深入的理解。最后,Darknet框架整体结构比较成熟完整,读者能在移植和优化过程中了解到优秀的机器学习框架的架构与编写方式,对软件架构能力的提升有一定帮助。

\section{Darknet源码结构}
从Joseph Redmon的个人网站或Github上下载得到的Darknet源码包解压后可得到如图所示的文件和文件夹,各文件和文件夹的作用如下\cite{darknet_note}:
\begin{itemize}
	\item cfg文件夹内是一些模型的架构,每个cfg文件类似于caffe的prototxt文件,通过该文件定义的整个模型的架构;
	\item data文件夹内放置了一些label文件,如coco9k的类别名等,和一些样例图,主要在训练和测试时使用;
	\item src文件夹内是最底层的框架定义文件,所有层的定义等最基本的函数全部在该文件夹内,是Darknet框架的源码所在;
	\item examples文件夹是更为高层的一些函数,如检测函数,识别函数等,这些函数直接调用了底层的函数,是对底层函数的封装;
	\item include文件夹,顾名思义,存放头文件的地方;
	\item python文件夹里是使用python对模型的调用方法,基本都在darknet.py中。要实现python的调用,还需要用到darknet的动态库libdarknet.so;
	\item scripts文件夹中是一些脚本,如下载coco数据集,将voc格式的数据集转换为训练所需格式的脚本等;
	\item 一系列和功能无关的license文件;
	\item 本章的重点Makefile文件,用于框架的编译和运行。
\end{itemize}

\section{Darknet的移植}
\subsection{Linux make和Makefile介绍其二}
在\ref{subsec:Makefile介绍}节中已经介绍过Makefile的文件结构和Linux make的基本使用方法,在本节中将进一步介绍Makefile文件中的变量定义和计算,为理解Darknet的编译过程打下基础。

在Makefile文件中,如果有某些指令和名称比较长或是需要在文件不同位置多次使用,为了保证可读性和修改的方便则需要用类似C语言宏定义的方式,用一些“宏”(更准确地说应该是变量)代表这些指令或名称。Linux make和Makefile即提供了这样的功能,这种功能的存在也使得Makefile被称为“make脚本”\footnote{为啥叫“脚本”?“脚本”的英文是“script”,可以联想两种解释型编程语言:js和python。js全称是“JavaScript”,python的源代码文件也叫“python script”,它们都是以解释器解释脚本文件中源代码的方式运行的。make也可以看作是一个解释型编程语言的解释器,Makefile就是它读入并解释的脚本文件,\ref{subsec:Makefile介绍}节和本节就是在介绍这个编程语言的语法}。这一功能可以概况为如下几种语法:
\begin{itemize}
	\item 变量定义:
	      \begin{lstlisting}
变量名=值;
    \end{lstlisting}
	\item 变量调用:
	      \begin{lstlisting}
$(变量名)
    \end{lstlisting}
	\item 变量修改:
	      \begin{lstlisting}
变量名+=$(变量名)或值
    \end{lstlisting}
	      \begin{lstlisting}
变量名=$(变量名)或值+$(变量名)或值+...
    \end{lstlisting}
	\item 变量比较和if语句:
	      \begin{lstlisting}
ifeq ($(变量名), $(变量名)或值)
[一些操作]
endif
    \end{lstlisting}
	\item 内嵌函数:
	      \begin{lstlisting}
$(函数名 变量1,变量2,...)
    \end{lstlisting}
	\item 静态模式:可以将多个任务合并于一个任务定义中,如下任务运行前{\codefont\$@}将被\code{文件路径\%.o}替代、{\codefont\$<}将被\code{文件路径\%.cpp}替代,生成每个.o和每个.cpp一一对应的多个指令:
	      \begin{lstlisting}
[文件路径]%.o: [文件路径]%.cpp [其他dependences]
    [一部分指令] $< [一部分指令] $@
    \end{lstlisting}
	      而下面这种的{\codefont\$@}将被\code{target}替代、{\codefont\$\^{}}被\code{dependences}替代,但没有上面那种多重替代的效果:
	      \begin{lstlisting}
[target]: [dependences]
    [一部分指令] $^ [一部分指令] $@
    \end{lstlisting}
\end{itemize}

Makefile中的所有变量均是字符串,变量修改中的“+”表示的就是字符串的连接,变量比较就是字符串的比较;字符串的调用可以在文件的任何地方,Linux make在运行时会自动将Makefile中的变量调用替换为变量对应的字符串\footnote{就像C语言的宏一样}。例如\ref{subsec:编写Makefile}中的Makefile:
\begin{lstlisting}
arrAdd:master_arrAdd.o slave_arrAdd.o
    sw5cc -hybrid master_arrAdd.o slave_arrAdd.o -o arrAdd
master_arrAdd.o:master_arrAdd.c
    sw5cc -host -c master_arrAdd.c
slave_arrAdd.o:slave_arrAdd.c
    sw5cc -slave -c slave_arrAdd.c
clean:
    -rm master_arrAdd.o slave_arrAdd.o arrAdd
run:arrAdd
    bsub -I -b -q q_sw_expr -n 1 -cgsp 64 ./arrAdd
\end{lstlisting}
把其中的输出文件名“arrAdd”、两个.o文件和bsub的运行选项用变量表示,可将Makefile改写如下:
\begin{lstlisting}
EXEC=arrAdd
OBJS=master_arrAdd.o slave_arrAdd.o

OPT=-I -b
OPT+=-q q_sw_expr
OPT+=-n 1
OPT+=-cgsp 64

$(EXEC):$(OBJS)
    sw5cc -hybrid $(OBJS) -o $(EXEC)
master_arrAdd.o:master_arrAdd.c
    sw5cc -host -c master_arrAdd.c
slave_arrAdd.o:slave_arrAdd.c
    sw5cc -slave -c slave_arrAdd.c
clean:
    -rm $(OBJS) $(EXEC)
run:$(EXEC)
    bsub $(OPT) ./$(EXEC)
\end{lstlisting}

静态模式语法可以以一个任务定义同时定义多个任务,例如当上面的Makefile要编译多个主核和从核源文件时,可以用内嵌函数和静态模式语法进一步缩写:
\begin{lstlisting}
MASTER=master/
SLAVE=slave/
OBJDIR=obj/

EXEC=arrAdd
OBJ=master_arrAdd1.o master_arrAdd2.o slave_arrAdd1.o slave_arrAdd2.o
OBJS=$(addprefix $(OBJDIR), $(OBJ))

OPT=-I -b
OPT+=-q q_sw_expr
OPT+=-n 1
OPT+=-cgsp 64

$(EXEC):$(OBJS)
    sw5cc -hybrid $^ -o $@
$(OUTDIR)%.o:$(MASTER)%.c
    sw5cc -host -c $< -o $@
$(OUTDIR)%.o:$(SLAVE)%.c
    sw5cc -slave -c $< -o $@
clean:
    -rm $(OBJS) $(OUTPUT)
run:$(EXEC)
    bsub $(OPT) ./$@
\end{lstlisting}
在此Makefile目录下运行Linux make时,行16$\sim$19的两个静态模式语法会使得make从MASTER变量和SLAVE变量所指目录下找到所有.c文件编译后放入OUTDIR所指目录中,即将多个任务用一个任务完成定义。在大型项目中善用此方法可以大大减少Makefile文件的长度,使项目更易于维护。

\subsection{Darknet的Makefile}
官网下载的Darknet源码中的Makefile文件如附录\ref{apdx:Darknet的Makefile}所示,本节将对这个Makefile文件及其所定义的编译过程进行解析。

\subsubsection{变量定义}
Makefile文件1$\sim$30行是变量的定义,其各部分作用如下:
\begin{itemize}
	\item 行1$\sim$5:整体的编译设定。指定编译过程中是否包含GPU、CUDNN、OpenCV、OpenMP和DEBUG模式。这些选项会在第32$\sim$59行被一系列ifeq调用,根据其值改变编译选项;
	\item 行7$\sim$12:NVCC的编译选项。这个变量在第92行编译.cu文件时调用。NVCC是Nvidia C Compiler,编译CUDA的.cu文件的编译器,这里的这些编译选项表示编译的目标设备的算力\footnote{对于不同算力的Nvidia卡NVCC有不同的优化策略};
	\item 行16$\sim$20:SLIB和ALIB是生成的链接库的.so和.a文件名称、EXEC是生成的可执行文件的名称、OBJDIR是中间文件的存储路径,它们都在后面的编译过程中被调用。VPATH是Makefile的一个特殊变量,VPATH中指代的目录下的文件在后面的过程中不需要再输完整路径\footnote{就像Windows和Linux里面的环境变量},多个目录以冒号分隔。VPATH实际发挥作用的地方是行85$\sim$92三个静态模式语法中没有路径的.c、.cpp、.cu源文件;
	\item 行22$\sim$30:CC、CPP、NVCC、AR都是GCC里的编译器,ARFLAGS、OPTS、LDFLAGS、COMMON、CFLAGS都是这些编译器的编译选项,这些变量也都在后面的编译过程中被调用。
\end{itemize}

\subsubsection{编译选项的设置}
Makefile文件32$\sim$73行是编译选项的设置,其主要作用有二:
\begin{itemize}
	\item 用ifeq根据行1$\sim$5定义的整体编译设定修改行22$\sim$30定义的编译选项;
	\item 确定输出文件的文件位置,以供make检查文件更新使用(行68和69的内嵌函数addprefix是加前缀,行70的\$(wildcard src/*.h)是返回src目录下的所有.h文件路径)。这些文件位置都会在后面的编译过程被调用。
\end{itemize}

\subsubsection{编译任务}
从行76开始的所有部分均是和编译相关的任务定义。其各部分作用如下:
\begin{itemize}
	\item 行72:默认make任务,生成obj、backup、results三个目录(行94$\sim$99的任务),生成静态链接库和动态链接库(行79$\sim$83 ALIB和SLIB变量值对应的任务),生成可执行文件(EXEC变量值对应的任务);
	\item 行76$\sim$77:生成可执行文件。该任务依赖于obj目录下的一系列.o文件(变量EXECOBJ)和静态链接库(变量ALIB);
	\item 行79$\sim$83:静态模式语法,用obj目录下的一系列.o文件(变量OBJS)生成静态链接库和动态链接库;
	\item 行85$\sim$92:静态模式语法,用源代码生成obj目录下一系列.o文件;
	\item 行94$\sim$104:编译相关的其他任务。.PHONY: clean使clean任务必被执行。
\end{itemize}

\subsection{开始移植Darknet}
做了这么多前期学习,终于可以正式开始移植Darknet了。但是有了前面那些分析感觉移植Darknet的这一节没啥好讲的了。移植主要是改Makefile文件,这里简单记录一下我改了哪吧。读者请自己下载Darknet在神威上编译,按照编译错误信息一步步进行修改,下面这个修改过程也是对着错误信息一步步出来的,仅供参考。

\subsubsection{Makefile文件修改}
\begin{enumerate}
	\item 编译器
	      首先要修改的肯定是编译器。要把原来的C语言编译器(变量CC)和C++编译器(变量CPP)的值改成神威里的编译器\code{sw5cc}和\code{sw5CC};
	\item 编译选项
	      其次是要改编译选项(变量CFLAGS)。原来的一些编译选项在\code{sw5cc}里面是不支持的,然后因为目前只是移植所以编译选项设置为\code{-host}全部在主核编译运行;
	\item 连接选项
	      在连接任务中,sw5cc连接时要加上\code{-hybrid}选项;
	\item 运行方式
	      原有的Makefile文件里面没有运行任务,要加一个运行任务(如\ref{sec:程序的运行}节);
	\item 其他修改
	      \begin{itemize}
		      \item 生成动态链接库有点问题,目前无法解决,故先删去生成动态链接库的任务;
		      \item 为了之后方便优化和维护继续往Makefile里加东西时不会和原来不支持的编译模式弄混,最好把所有神威系统不支持的编译模式删掉,包括CUDA、OpenMP和OpenCV编译,只保留没有任何依赖的普通模式和debug编译模式。
	      \end{itemize}
\end{enumerate}

移植后Darknet的Makefile文件如附录\ref{apdx:移植后Darknet的Makefile文件}所示。

\subsubsection{源代码文件修改}
\begin{enumerate}
	\item include/darknet.h
	      这个文件在编译时报类型重定义错误,原因是里面的network和layer类型嵌套方式的问题,改成能正常编译的struct嵌套就行了。
\end{enumerate}

经过以上修改,即可在Darknet根目录下运行“make”生成可执行文件、“make run”提交运行了。运行“make run”若任务队列返回了“usage: ./darknet <function>”即说明移植成功。

\section{本章总结}
本章的主要脉络如下:
\begin{enumerate}
	\item 解析Darknet源码文件的结构;
	\item 进一步讲解了Makefile文件的语法,为解析Darknet的Makefile做铺垫;
	\item 解析Darknet的Makefile,明确移植Darknet需要修改的地方;
	\item 移植Darknet,解决一些细节问题。
\end{enumerate}
